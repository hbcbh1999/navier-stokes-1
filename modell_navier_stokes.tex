%\datum{13. April 2015}
\section{Die Navier-Stokes-Gleichungen als Modell für inkompressible Strömungen}
\begin{bemerkung}
  Grundprinzipien und Variablen

Die Grundgleichungen der Fluiddynamik werden \markdef{Navier-Stokes-Gleichungen} genannt. Im isothermen Fall, das heißt, eine Strömung konstanter Temperatur, betrachten wir zwei physikalische Bilanzgleichungen: die Masseerhalung und die Impulsbilanz /Impulserhaltung. Es gibt verschiedene Wege, diese Gleichungen herzuleiten. Im Folgenden wird ein klassischer Zugang aus der Kontinuumsmechanik präsentiert. In der Herleitung sind einige Schritte phänomenologisch/heuristisch begründet. 

Die Strömung wird durch folgende Variablen beschrieben:
\begin{itemize}
\item $\rho(t, x)$: Dichte [kg/m$^{3}$]
\item $v(t, x)$: Geschwindigkeit [m/s]
\item $p(t, x)$: Druck [Pascal: Pa = N/m$^{2}$].
\end{itemize}
Diese werden als genügend glatte Funktionen angenommen. Das \markdef{Gebiet} wird $\Omega \subset \R^{3}$ genannt. Die zeitliche Evolution wird im Zeitintervall $[0, T]$ modelliert. 
\end{bemerkung}

\subsection{Die Massenerhaltung}
\begin{bemerkung} Ein Erhaltungssatz

Sei $\omega$ ein beliebiges, offenes Teilgebiet von $\Omega$ mit genügend glatter Oberfläche $\partial \omega$, das konstant in der Zeit ist. Dann ist die Gesamtmasse in $\omega$ gegeben durch
\begin{align*}
  m(t) = \int_{\omega}\rho(t, x) \,dx
\end{align*}
(in [kg]).

Wenn die Masse in der zeitlichen Evolution in $\omega$ erhalten bleibt, so muss die zeitliche Rate der Massenänderung mit dem Massenfluss $\rho v(t, x)$ [kg/m$^{3}$s] über die Oberfläche von $\partial \omega$ übereinstimmen. Das heißt, die Änderung der Masse beträgt
\begin{align}\label{eq:mass_deriv}
  \frac{d}{dt} m(t) = \frac d {dt} \int_{\omega} \rho(t, x) dx = - \int_{\pd \omega} \rho v(t, x) \cdot n \,ds,
\end{align}
wobei $n(s)$ eine äußere Einheitsnormale bei $s \in \pd \omega$ ist. Da alle Funktionen als genügend glatt angenommen werden, kann der Gauß'sche Satz angewendet werden. 

Mit \eqref{eq:mass_deriv} führt dies auf
\begin{align*}
  \int_{\omega} \pd_{t}\rho(t, x) + \nabla\cdot(\rho v)\,dx = 0. 
\end{align*}
Da $\omega$ ein beliebiges Kontrollvolumen ist, folgt
\begin{align}\label{eq:continuous_mass}
  \pd_{t} \rho + \nabla \cdot(\rho v) = 0, \qquad \forall t \in [0, T], \, x \in \Omega
\end{align}
(schöne Integralgleichung).
Diese Erhaltungsgleichung ist die erste Grundgleichung der Fluiddynamik., die (Massen-)Kontinuitätsgleichung.  
\end{bemerkung}
\begin{bemerkung} Inkompressible, homogene Fluide

Wenn das Fluid inkompressibel ist (Flüssigkeit) und homogen, das heißt, es besteht aus einer Komponente, dann, wird angenommen, gilt approximativ
\begin{align*}
  \rho(t, x)= \rho = c
\end{align*}
und \eqref{eq:continuous_mass} reduziert sich auf
\begin{align*}
  &0 = \rho_{t} + \nabla\cdot(\rho v) =  \rho \cdot \nabla v + v \cdot \nabla \rho = \rho \nabla \cdot v\\
\Rightarrow \quad&\nabla \cdot v = 0.
\end{align*}
Es wird angenommen, dass
\begin{align*}
  v(t, x) =
  \begin{pmatrix}
    v_{1}(t, x)\\
    v_{2}(t, x)\\
    v_{3}(t, x)
  \end{pmatrix}
\end{align*}
und es gilt
\begin{align*}
  \nabla \cdot v = \pd_{x} v_{1} + \pd_{y} v_{2} + \pd_{z} v_{3}.
\end{align*}
Daher reduziert sich im inkompressiblen Fall die Massenerhaltung in einem homogenen, imkompressiblen Fluid auf eine (geometrische) Zwangsbedingung an das Geschwindigkeitsfeld. 

%Mit 'geometrisch' ist gemeint: in jedem Punkt muss genau das rausfließen, was reinfließt. 
\end{bemerkung}
\subsection{Die Bilanzgleichung / Erhaltungsgleichungen für den Impuls}
\begin{bemerkung} Newtons zweite Bewegungsgleichung
  
Die Bilanzgleichung für den Impuls folgt aus Newtons Bewegungsgleichung
\begin{align}\label{eq:n2}
  \text{Kraft} = \text{Masse} \cdot \text {Beschleunigung}.
\end{align}
Diese Gleichung sagt aus, dass die Änderungsrate des Impulses gleich der Gesamtkraft auf die Fluidpartikel in einem Kontrollvolumen ist. 
\end{bemerkung}
\begin{bemerkung} Beschleunigung
  
Man betrachtet ein Fluidpartikel zur Zeit $t$ am Ort $x$ mit geschwindikeit $v(t, x)$ in einem Zeitintervall $\Delta t$. Eine lineare Extrapolation des Partikelpfades ergibt, dass sich das Teilchen zur Zeit $(t + \Delta t)$ sich am Ort $(x + \Delta t v)$ befindet.

Die Beschleunigung des Teilchens ergibt sich zu
\begin{align*}
  \frac{D v}{\Delta t}(t, x) &= \lim_{t \to 0}\frac{v(t + \Delta t, x + \Delta t v (t, x)) - v(t, x)}{\Delta t}\\
&\approx \lim_{\Delta t \to 0} \frac{v(t, x)+ \Delta t \cdot \pd_{t}v(t, x) + \nabla v(t, x)(\Delta t \cdot v) - v(t, x)}{\Delta t}\\
&=\pd_{t} v (t, x) + \nabla v(t, x)\cdot v(t, x)
\end{align*}
Hier wird mit $\nabla v$ bezeichnet:
\begin{align*}
  \nabla v =
  \begin{pmatrix}
    \pd_{x} v_{1} &     \pd_{y} v_{1} &     \pd_{z} v_{1} \\
    \pd_{x} v_{2} &     \pd_{y} v_{2} &     \pd_{z} v_{2} \\
    \pd_{x} v_{3} &     \pd_{y} v_{3} &     \pd_{z} v_{3} 
  \end{pmatrix}.
\end{align*}
In der Fluiddynamik wird der Term $(\nabla v)\cdot v$ oft geschrieben als $(v \cdot \nabla) v$ oder $v \cdot \nabla v$. Gemeint ist jedenfalls immer
\begin{align*}
  (\nabla v)\cdot v =    \begin{pmatrix}
    v_{1}\pd_{x} v_{1} &     v_{2}\pd_{y} v_{1} &     v_{3}\pd_{z} v_{1} \\
    v_{1}\pd_{x} v_{2} &     v_{2}\pd_{y} v_{2} &     v_{3}\pd_{z} v_{2} \\
    v_{1}\pd_{x} v_{3} &     v_{2}\pd_{y} v_{3} &     v_{3}\pd_{z} v_{3} 
  \end{pmatrix}.
\end{align*}
Bei der Notation $(v \cdot\nabla) = v_{1} \pd_{x} + v_{2} \pd_{y} + v_{3} \pd_{z}$ denkt man an einen Operator, der auf jede einzelne Komponente von $v$ wirkt. 

Der Gradient der Geschwindigkeit wird auch gerne als Tensor geschrieben:
\begin{align*}
  (\nabla v)_{ij} = \pd_{j} v_{i}, \qquad i, j = 1, 2, 3.
\end{align*}
Letztendlich wird eine Approximation erster Ordnung benutzt, um die Beziehung 'Masse $\cdot$ Beschleunigung' in beliebigen Kontrollvolumen zu modellieren.
\begin{align*}
  \text{'Masse} \cdot \text{Beschleunigung'} = \int_{\omega} \rho(t, x) (\pd_{t} v + (v \cdot \nabla) v (t, x)) v (t, x) dx
\end{align*}
(in [N]). Dieser Ausdruck muss noch mit der Gesamtkraft bilanziert werden, die  auf $\omega$ wirkt, gemäß der zweiten Bewegungsgleichung von Newton. 
\end{bemerkung}
%\datum{15. April 2015}
\begin{bemerkung}
  Alternative Ableitung: Der Impuls im Kontrollvolumen $\omega$ ist gegeben durch
  \begin{align*}
    \int_{\omega} \rho v (t, x) dx \quad [Ns].
  \end{align*}
Dann kann die Bilanzgleichung für den Impuls analog formuliert werden wie die Masseerhaltungen \eqref{eq:mass_deriv}:
\begin{align*}
\int_{\omega} (\rho v)_{t} dx =  \frac d {dt} \int_{\omega} \rho v (t, x) dx = - \int_{ \pd \omega}(\rho v)(v \cdot n)(t, s) ds + \int_{\omega} f_{net}(t, x) dx \quad [N]
\end{align*}
Hierbei bezeichnet $\frac d {dt} \int_{\omega} \rho v(t, x) dx$ die Änderungsraten des Impulses und der erste Term auf der rechten Seite modelliert den Impulsfluss über die Oberfläche von $\omega$ (im Strömungsfeld $v$). $f_{net}$ [N/m$^{3}$] bezeichnet eine sogenannte Kraftdichte. $\int_{\omega} f_{net}(t, x)dx$ bezeichnet dann die Gesamtkraft auf das Kontrollvolumen $\omega$. Weiter gilt:
\begin{align*}
  v (v \cdot n) = v v^{T}n.
\end{align*}
Partielle Integration und Vertauschen von Integration und Differentiation ergibt
\begin{align*}
  \int_{\omega} \frac d {dt} (\rho v) + \nabla\cdot(\rho v v ^{T})(t, x) dx = \int_{\omega} f_{net}(t, x)dx.
\end{align*}
Dabei wird Folgendes verwendet:
\begin{itemize}
\item Das dyadische Produkt ist
  \begin{align*}
    v v ^{T} =
    \begin{pmatrix}
      v_{1}v_{1} &       v_{1}v_{2} &       v_{1}v_{3} \\
      v_{2}v_{1} &       v_{2}v_{2} &       v_{2}v_{3} \\
      v_{3}v_{1} &       v_{3}v_{2} &       v_{3}v_{3} 
    \end{pmatrix}
 \eqqcolon v \otimes v.
  \end{align*}
\item Die Tensordivergenz
  \begin{align*}
    \nabla \cdot
    \begin{pmatrix}
      a_{11} &       a_{12} &      a_{13} \\
      a_{21} &       a_{22} &      a_{23} \\
      a_{31} &       a_{32} &      a_{33} 
    \end{pmatrix}
=
\begin{pmatrix}
  \nabla \cdot
  \begin{pmatrix}
    a_{11} \\ a_{12}\\ a_{13}
  \end{pmatrix}\\
  \nabla \cdot
  \begin{pmatrix}
    a_{21} \\ a_{22}\\ a_{23}
  \end{pmatrix}\\
  \nabla \cdot
  \begin{pmatrix}
    a_{31} \\ a_{32}\\ a_{33}
  \end{pmatrix}
\end{pmatrix}
  \end{align*}
\end{itemize}
Die Produktregel ergibt
\begin{align*}
  \int_{\omega} \left( (\pd_{t} \rho) v + \rho \pd_{t}v + v v^{T} \nabla \rho + \rho(\nabla \cdot v) v + \rho(v \cdot \nabla) v\right) dx = \int_{\omega} f_{net}(t, x) dx.
\end{align*}
Es gilt nun
\begin{align*}
&  (\pd_{t} \rho)v + v v^{T}\nabla \rho + \rho(\nabla v)v = v (\pd_{t}\rho + v \cdot \nabla \rho + \rho \nabla \cdot v) = v \underbrace{(\pd_{t} \rho + \nabla \cdot (\rho v))}_{= 0}\\
&\int_{\omega}(\rho \pd_{t} v + \rho(v \cdot \nabla) v)(t, x) dx = \int_{\omega} f_{net} (t, x) dx.
\end{align*}
Da $\omega$ als beliebig angenommen wurde, gilt die Bilanzgleichung 
\begin{align*}
  \rho \pd_{t}v + \rho(v \cdot \nabla)v = f_{net} \qquad \forall t \in [0, T], x \in \Omega.
\end{align*}
\end{bemerkung}
\begin{bemerkung} Äußere Kräfte 

Die Kräfte, die auf $\omega$ wirken, sind zusammengesetzt aus sogenannten \markdef{Volumen-/äußeren Kräften} und \markdef{inneren Kräften}. Äußere Kräfte sind zum Beispiel Schwerkraft, elektromagnetische Kräfte (in Flüssigmetallen), Corioliskraft. Diese Kräfte werden modelliert durch die Volumenkraft
\begin{align*}
  \int_{\omega} f_{ext}(t, x), \quad f_{ext}:\,[N/m^{3}].
\end{align*}  
\end{bemerkung}
\begin{bemerkung} Cauchy'sche Spannungsprinzip und der Spannungstensor

Innere Kräfte sind Kräfte, die das Fluid auf sich selbst ausübt. Diese Kräfte beinhalten den Druck und den Reibungswiderstand, den ein Flächenelement auf ein benachbartes Flächenelement auswirkt.
Nach dem Cauchy'schen Spannungsprinzip werden die inneren Kräfte des Fluids als Kontraktionskräfte axiomatisch modelliert. Das bedeutet, es wird angenommen, dass sie an der Oberfläche $\pd \omega$ des Fluidelements angreifen. Es bezeichne $\vec t$ [N/m$^{2}$] diesen inneren Kraftvektor, der \markdef{Cauchy'sche Spannunsvektor} oder \markdef{Schnittspannungsvektor}. Dann ist der Beitrag der inneren Kräfte gegeben durch
\begin{align*}
  \int_{\pd \omega} \vec t (t, s) ds
\end{align*}
Addiert man die äußeren und inneren Kräfte, so erhält man als Bilanzgleichung für den Impuls für ein beliebiges, in der Zeit konstantes Kontrollvolumen
\begin{align}\label{eq:bilanz}
  \int_{\omega} \rho(\pd_{t} v) + (v \cdot \nabla) v dx = \int_{\omega} f_{ext} dx + \int_{\pd \omega} \vec t (t, s) ds.
\end{align}
Um eine partielle Differentialgleichung herleiten zu können, muss man $\int_{\pd \omega} \vec t(t, s) ds$ als \markdef{Volumenintegral} ausdrücken können. Dies wird im Folgenden hergeleitet. Dabei bezeichnet die rechte Seite von \eqref{eq:bilanz} die Gesamtkraft, die auf $\omega$ und $\pd \omega$ angreift. 

Grundlage ist das Cauchy'sche Spannungsprinzip, das innere Kräfte als Kontakkräfte via
\begin{align*}
  \int_{\pd \omega} \vec t (t, s) ds
\end{align*}
modelliert. An jeder gedachten Ebene an $\pd \omega$ greift eine Kraft an, die \emph{geometrisch} nur von der Orientierung der Ebene, dem Ort $x \in \pd \omega$ und dem Zeitpunkt $t$ abhängt. Es gilt also $\vec t = \vec t (t, x, n)$. Der Schnittspannungsvektor hängt also zum Beispiel von der Krümmung von $\pd \omega$ aboder von Kräften, die außerhalb von $\pd \omega$ liegen, zum Beispiel Fernwirkung wie bei elektrischen Ladungen. Dabei bezeichnet $n$ eine äußere Normale der Tangentialebene an $\pd \omega$.  
\end{bemerkung}
Im Folgenden wird die grundlegende Überlegung von Cauchy modelliert. In jedem Punkt $x \in \pd \omega$ sind die Schnittspannungsvektoren zu verschiedenen gedachten Ebenen Linearkombinationen von einigen ausgezeichneten Richtungen, das heißt, sie lassen sich beschreiben als
\begin{align*}
  \vec t = \sigma^{T} \cdot n.
\end{align*}
Der Spannungstensor $\sigma^{T}$ ist eine Matrix und ist von der Orientierung der Schnittflächen unabhängig. Manchmal schreibt man, dass $\sigma$ eine lineare Abbildung vermittelt. Allerdings ist die Abbildung $\vec t = \vec t (n)$ natürlich nicht linear, da die Summe zweier Normalen normal ist. 
Die Überlegungen fußen auf Cauchy's Tetraederargument:

Sei $\omega$ ein Tetraeder mit Eckpunkten $p_{0} = (0, 0, 0)^{T}$, $p_{1} = (x_{1}, 0, 0)^{T}$, $p_{2} = (0, y_{2}, 0)$, $p_{3} = (0, 0, z_{3})$. Die Ebene, die von $p_{1}, p_{2}, p_{3}$ aufgespannt wird, wird mit $\pd \omega^{(n)}$ bezeichnet. Die äußere Normale von $\pd \omega^{(n)}$ ist gegeben durch
\begin{align*}
  n = \frac{(p_{2} - p_{1}) \times (p_{3} - p_{1})}{\nnorm{(p_{2} - p_{1}) \times (p_{3} - p_{1})}}, 
\end{align*}
und es gilt
\begin{align*}
  (p_{2} - p_{1}) \times (p_{3} - p_{1}) =
  \begin{pmatrix}
    -x_{1} \\ y_{2} \\ 0 
  \end{pmatrix}
\times 
\begin{pmatrix}
    -x_{1} \\ 0 \\ z_{3} 
  \end{pmatrix}
 = 
\begin{pmatrix}
    y_{2}z_{3} \\ x_{1}z_{3} \\ x_{1}y_{2} 
  \end{pmatrix}.
\end{align*}
Dieser Vektor zeigt von $p_{0}$ weg, $n$ ist also eine äußere Normale von $n = (n_{1}, n_{2}, n_{3})^{T}$.
%\datum{20. April 2015}

%\datum{27. April 2015}
\setcounter{satz}{13}
\begin{bemerkung}
  \begin{align*}
    \V = a\D(v) + b(\nabla\cdot v) I
  \end{align*}
Diese lineare Beziehung ist nur eine Approximation an ein reelles Fluid. Im Allgemeinen wird die Bedingung nichtlinear sein. Nur für kleine Spannungen kann eine lineare Spannungs-Deformations-Beziehung verwendet werden. In solchen Fällen spricht man von \markdef{Newtonischen Fluiden}. (Nicht-Newtonische Fluide: z.B. Zahnpasta, Blut) 
\end{bemerkung}
\begin{bemerkung}Normal- und Scherspannung, Spur des Spannungstensors. 

Die Diagonalkomponenten $\sigma_{11}, \sigma_{22}, \sigma_{33}$ des Spannungstensors werden \markdef{Normalspannung} und die Nebendiagonalelemente \markdef{Scherspannung} genannt. Für inkompressible Fluide gilt:
\begin{align}\label{eq:spannungstensor_inkompressibel}
  \sigma = 2 \mu \D (v) - PI.
\end{align}
Die Spur des Spannungstensors ist die Summe der Normalspannung
\begin{align*}
  \spur(\sigma) &= \sigma_{11} + \sigma_{22} + \sigma_{33}\\
& = 2 \mu (\partial_{x} v_{1} + \partial_{y} v_{2} + \partial_{z} v_{3}) + 3(\zeta - \frac{2\mu} 3)(\nabla\cdot v) - 3 P\\
&= 3 \zeta (\nabla \cdot v) - 3 P. 
\end{align*}
Für inkompressible Fluide gilt:
\begin{align*}
  P(t, x) = - \frac 1 3(\sigma_{11} + \sigma_{22} + \sigma_{33}).
\end{align*}


\begin{align*}
(  \V = 2 \mu \D(v) + (\zeta - \frac{2 \mu} 3)I)
\end{align*}

\end{bemerkung}
\begin{bemerkung}
  Die Navier-Stokes-Gleichungen
  \begin{align}\label{eq:ns}
    \rho(\partial_{t} v + (v \cdot \nabla)v - 2 \nabla \cdot (\mu \D (v))) - \nabla \cdot ((\zeta - \frac {2 \mu} 3)(\nabla \cdot v)I) + \nabla p &= f_{\ext} \qquad (0, T] \times \Omega\\
\partial_{t} \rho + \nabla\cdot (\rho v) &= 0 \qquad (0, T) \times \Omega
  \end{align}
Wenn das Fluid inkompressibel und homogen ist mit positiven Konstanten $\mu$ und $\rho$, erhält man die inkompressiblen Navier-Stokes-Gleichungen
\begin{align}\label{eq:ns_inc}
    \partial_{t} v  - 2  v \nabla \cdot (\D (v)) + (v \cdot \nabla)v + \nabla   \left( \frac p \zeta \right) &= \frac{f_{\ext}} \rho \qquad (0, T] \times \Omega\\
\nabla \cdot v &= 0 \qquad (0, T] \times \Omega
\end{align}
Hier ist $v = \frac \mu \rho [m^{2}/s] $ die sogenannte kinematische Viskosität des Fluids. 
\end{bemerkung}

\subsection{Die entdimensionalisierten Navier-Stokes-Gleichungen}
\label{sec:entd-navi}

\begin{bemerkung} Charakteristische Skalen
  
Die mathematische Analysis und numerische Simulation basieren auf entdimensionalisierten (dimensionslosen) Gleichungen, die aus \eqref{eq:ns_inc} hergeleitet werden. Dazu werden folgende Größen eingeführt:
\begin{itemize}
\item $L [m]$: eine charakteristische Längenskala des Problems
\item $U[m/s]$: eine charakteristische Geschwindigkeit
\item $T^{*}[s]$: eine charakteristische Zeitskala
\end{itemize}
\end{bemerkung}
\begin{bemerkung}
  Die Navier-Stokes-Gleichung in entdimensionalisierter Form 

Bezeichne $(t', x')[s, m]$ die alten Variablen. Wende die Transformation
\begin{align}\label{eq:var_trafo}
  x = \frac {x'} L, \quad u = \frac v U, \quad t = \frac {t'} {T^{*}} 
\end{align}
an, und man erhält aus \eqref{eq:ns_inc} und einer Reskalierung 
\begin{align*}
    \frac L {U T^{*}} \partial_{t} u  - \frac{2  \nu}{u\cdot L} \nabla \cdot (\D (u)) + (u \cdot \nabla)u + \nabla   \left( \frac P {\zeta U^{2}} \right) &= \frac L{\rho U^{2}}f_{\ext}  \qquad (0, T] \times \Omega\\
\nabla \cdot u &= 0 \qquad (0, T] \times \Omega
\end{align*}
Auch das Gebiet und das Zeitintervall sind entdimensionalisiert. mit der Definition
\begin{align}\label{eq:pressure}
  p = \frac P {\rho  U^{2}}, \quad \Rey = \frac {U\cdot L}{ v}, \quad \St = \frac L {U T^{*}}, \quad f = \frac L {\rho  U^{2}} f_{\ext}
\end{align}
erhält man die Navier-Stokes-Gleichungen in entdimensionalisierter Form
\begin{align}\label{eq:ns_nodim}
      \St\partial_{t} u  - \frac2  \Rey \nabla \cdot (\D (u)) + (u \cdot \nabla)u + \nabla p&= f  \qquad (0, T] \times \Omega\\
\nabla \cdot u &= 0 \qquad (0, T] \times \Omega
\end{align}
Die Konstanten $\Rey$ und $\St$ werden Reynoldszahl und Strouhalszahl genannt. Die Zahlen erlauben die Klassifizierung und den Vergleich und Strömungen. 
\end{bemerkung}
\begin{bemerkung}Inhärente Schwierigkeiten der Navier-Stokes-Gleichungen

Um die Notation zu vereinfachen, setzt man die charakteristische Zeit auf $T^{*} = \frac L U$, sodass sich \eqref{eq:ns_nodim} vereinfacht zu
\begin{align}\label{eq:ns_nodim_simple}
  \partial_{t} u - 2 \nu \nabla \cdot (\D(v)) + (u \cdot \nabla) u + \nabla p &= f, \qquad (0, T] \times \Omega\\
\nabla \cdot u &= 0, \qquad (0, T] \times \Omega
\end{align}
mit der dimensionslosen Konstante $\nu = \frac 1 \Rey$.  \eqref{eq:ns_nodim_simple} enthält folgende Schwierigkeiten:
\begin{itemize}
\item die Koppelung von Geschwindigkeit und Druck (zweite Gleichung definiert Druck wesentlich), 
\item die Nichtlinearität des konvektiven Terms $(u \cdot \nabla) u$, 
\item die Dominanz des konvektiven Terms gegenüber dem viskosen Terms, wenn $\nu$ klein ist. 
\end{itemize}
\end{bemerkung}

\begin{bemerkung} Verschiedene Formen der Terme in \eqref{eq:ns_nodim_simple} 
  
Mithilfe von $\nabla \cdot u = 0$ wird der Reibungsterm vereinfacht:
\begin{align*}
&  \nabla \cdot \nabla u = \Delta u, \quad \nabla \cdot (\nabla u^{T}) = \nabla \cdot
  \begin{pmatrix}
    \partial_{x} u_{1} & \partial_{x} u_{2} & \partial_{x} u_{3} \\
    \partial_{y} u_{1} & \partial_{y} u_{2} & \partial_{y} u_{3} \\
    \partial_{z} u_{1} & \partial_{z} u_{2} & \partial_{z} u_{3} 
  \end{pmatrix}
 =
 \begin{pmatrix}
       \partial_{x} (\nabla \cdot u) \\
       \partial_{y} (\nabla \cdot u) \\
       \partial_{z} (\nabla \cdot u) 
 \end{pmatrix}
=
\begin{pmatrix}
  0\\0 \\0
\end{pmatrix}
\\
\implies& - 2 \nu \nabla \cdot (\D(v)) = - 2 \nu \nabla \cdot \left( \frac 1 2 (\nabla u) + \frac 1 2 (\nabla u)^{T}\right) = - \nu \Delta u.
\end{align*}
Weiter gilt $(u \cdot \nabla) u = \nabla \cdot (u \otimes u)$, wenn $ \nabla \cdot u= 0$. In vielen Diskretisierungen wird die Divergenzbedingung relaxiert, Vorsicht!
\end{bemerkung}
\begin{bemerkung}
  In einer stationären Strömung ändern sich Geschwindigkeit und Druck nicht in der Zeit. Dann gilt $\partial_{t} u = 0$. 
Das ergibt die stationären Navier-Stokes-Gleichungen:
\begin{align}\label{eq:ns_stat}
  - \nu \Delta u + (u \cdot \nabla) u + \nabla p &= f, \quad \Omega\\
\nabla \cdot u & = 0, \quad \Omega.
\end{align}
Notwendig für die Stationarität der Lösung ist, dass die Daten der Zeit (rechte Seite, Randbedingung) zeitunabhängig sind. Diese Bedingung ist aber nicht hinreichend (z.B. Kanalströmung mit großer Reynoldszahl). 
\end{bemerkung}

\begin{itemize}
\item Wenn in einer stationären Strömung der viskose Impulstransport dem Konvektiven überwiegt (im Falle sehr kleiner Reynoldszahlen), dann kann der nichtlineare Term $(u \cdot \nabla) u$ vernachlässigt werden. Dies ergibt das lineare Stokes-Modell
\begin{align}\label{eq:stokes}
  - \Delta u + \nabla p &= f, \quad \Omega\\
\nabla \cdot u &= 0, \quad \Omega
\end{align}
per Division durch $\nu$ und Neudefinition von $p$ und $f$.
\item In der Numerik werden oft die Oseen-Gleichungen betrachtet. Dazu sei $u_{0}$ gegeben mit $\nabla \cdot u_{0} = 0$. Dann
  \begin{align*}
    - \nu \Delta u + (u_{0} \cdot \nabla) u + c \cdot u + \nabla p &= f, \quad \Omega\\
\nabla \cdot u &= 0,\quad \Omega
  \end{align*}
mit einer positiven Konstanten $c > 0$. $c \cdot u$ kommt von einer Zeitdiskretisierung von $u_{t}$, $u_{t} = \frac {u^{n+1} - u^{n}}{\Delta t}$. 
\end{itemize}
%\datum{28. April 2015}

\subsection{Anfangs- und Randwertbedingungen}
\label{sec:anfangs-und-randw}

\begin{bemerkung}
  Die Navier-Stokes-Gleichungen sind partielle Differentialgleichungen erster Ordnung in der Zeit und zweiter Ordnung im Ort. Deshalb benötigt man Anfangsbedingungen zur Zeit $t = 0$ und Raundbedingungen für $\Gamma = \pd \Omega$, wenn $\Omega$ ein beschränktes Gebiet ist. Nur entdimensionalisierte Randbedingungen werden betrachtet. 
\end{bemerkung}
\begin{bemerkung}
  Anfangsbedingungen:
  \begin{align*}
    u(0, x) = u_{0}(x), \quad \nabla \cdot u_{0} = 0 \text{ in } \Omega
  \end{align*}
\end{bemerkung}
\begin{bemerkung}Dirichlet-Randbedingung, No-Slip-Bedingung, wesentliche Randbedingung.

Oft benutzt man
\begin{align*}
  u(t, x) = g(t, x) \quad \text{in } (0, T] \times \Gamma_{dir}
\end{align*}
  mit $\Gamma_{dir} \subset \Gamma$. Diese Randbedingungen werden \markdef{Dirichlet-Randbedingungen} genannt. Insbesondere Ein- und Ausströmungen können damit modelliert werden. Im Spezialfall $g(t, x) = 0$ spricht man von \markdef{No-Slip-Randbedingungen}. Sei $n$ eine Einheitsnormale an $x \in \Gamma_{nosl}\subset \Gamma_{dir}$ und seien $(t_{1}, t_{2})$ Einheitstangentialvektoren. Dann kann die No-Slip-Bedingung zerlegt werden in
  \begin{align*}
    u(t, x)\cdot n = 0, \quad u(t,x)\cdot t_{1} = 0, \quad u(t, x)\cdot t_{2} = 0
  \end{align*}
('keine Durchdringung der Wand', 'No-Slip', 'No-Slip'). Sind auf dem ganzen Rand Dirichlet-Randbedingungen gegeben, dann sind zwei Dinge zu beachten:
\begin{itemize}
\item Der Druck ist nur auf einer additiven Konstanten bestimmt, die man festlegen muss, z.B.:
  \begin{align*}
&    \int_{\Omega} p(t, x) dx = 0, \quad t \in (0, T]\\
& u_{t} - \nu \nabla u + (u \cdot \nabla) u + \nabla p = f, \quad \nabla\cdot u = 0
  \end{align*}
\item Kompatibilitätsbedingung:
  \begin{align*}
    0 = \int_{\Omega} \nabla \cdot u dx = \int_{\Gamma} u \cdot n dS = \int_{\Gamma} g \cdot n dS
  \end{align*}
\end{itemize}
Dirichlet-Randbedingungen sind für Stokes und Navier-Stokes sogenannte \markdef{wesentliche Randbedingungen}, die in die Definition des zu wählenden Funktionenraumes eingehen müssen. 
\end{bemerkung}
\begin{bemerkung}Free-Slip-Randbedingungen, Slip mit Reibung
  
Die \markdef{Free-Slip-Bedingung} wird angewendet für Ränder ohne Reibung, z.B. freie Oberfläche (Meeresoberfläche). 
Sie haben die Form
\begin{align}\label{eq:free_slip}
&  u \cdot n = g \quad \text{in } (0, T] \times \Gamma_{slip}, \notag \\
&n^{T}\cdot\sigma \cdot t_{K} = 0 \quad \text{in }(0, T] \times \Gamma_{slip}, 1\leq K \leq d-1
\end{align}
Für $g = 0$ gibt es keine Durchdringung der Wand. 'Slip mit linearer Reibung, keine Durchdringung':
\begin{align} \label{eq:slfr}
&  u \cdot n = 0\quad \text{in } (0, T] \times \Gamma_{slfr} \subset \Gamma,\notag \\
& u \cdot t_{n} + \beta^{-1}n^{T}\sigma t_{n} = 0\quad \text{in } (0, T] \times \Gamma_{slfr}, 1 \leq k\leq d-1
\end{align}
Dies bedeutet 'keine Durchdringung' und Reibung proportional und entgegengesetzt zur Tangentialgeschwindigkeit. Für $\beta^{-1} \to 0$ erhält man No-Slip-Bedingungen, für $\beta^{-1}\to \infty$ erhält man Free-Slip-Bedingungen. Die Schwierigkeit besteht in der experimentellen Bestimmung von $\beta$. 

Da $n$ und $t_{k}$ orthogonale Vektoren sind, sind \eqref{eq:free_slip} und \eqref{eq:slfr} druck-unabhängig. Man braucht eine additive Konstante für den Druck. 
\end{bemerkung}
\begin{bemerkung}
  Ausströmungs- und 'Do-Nothing'-Randbedingungen, natürliche Randbedingungen

  \begin{align}\label{eq:auff}
    \sigma \cdot n = 0 \quad \text{in }(0,T] \times \Gamma_{auff} \subset \Gamma 
  \end{align}
Mathematisch sind dies sogenannte \markdef{natürliche Randbedingungen}, da die Oberflächenintegrale am Rand verschwinden. \eqref{eq:auff} fixiert den Druck.  
\end{bemerkung}
\begin{bemerkung}
  Bedingungen für unendliche Gebiete und periodische Randbedingungen . 

Der Fall $\Omega = \R^{3}$ wird mathematisch oft betrachtet. Zwei Fälle kommen vor: 
\enu{\alph}
\begin{enumerate}
\item Vorschreiben des Abfalls des Geschwindigkeitsfeldes in Unendlichen für $\nnorm{x}_{2} \to \infty$. 
\item Periodische Randbedingungen, z.B. Würfel $(0, l)^{d}$ und
  \begin{align*}
    u(t, x + l\cdot e_{i}) = u(t, x), \quad \forall(t, x) \in (0, T] \times \Gamma.
  \end{align*}

\end{enumerate}
\end{bemerkung}
