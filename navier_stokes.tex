%\datum{13. April 2015}
\section{Die Navier-Stokes-Gleichungen als Modell für inkompressible Strömungen}
\begin{bemerkung}
  Grundprinzipien und Variablen

Die Grundgleichungen der Fluiddynamik werden \markdef{Navier-Stokes-Gleichungen} genannt. Im isothermen Fall, das heißt, eine Strömung konstanter Temperatur, betrachten wir zwei physikalische Bilanzgleichungen: die Masseerhalung und die Impulsbilanz /Impulserhaltung. Es gibt verschiedene Wege, diese Gleichungen herzuleiten. Im Folgenden wird ein klassischer Zugang aus der Kontinuumsmechanik präsentiert. In der Herleitung sind einige Schritte phänomenologisch/heuristisch begründet. 

Die Strömung wird durch folgende Variablen beschrieben:
\begin{itemize}
\item $\rho(t, x)$: Dichte [kg/m$^{3}$]
\item $v(t, x)$: Geschwindigkeit [m/s]
\item $p(t, x)$: Druck [Pascal: Pa = N/m$^{2}$].
\end{itemize}
Diese werden als genügend glatte Funktionen angenommen. Das \markdef{Gebiet} wird $\Omega \subset \R^{3}$ genannt. Die zeitliche Evolution wird im Zeitintervall $[0, T]$ modelliert. 
\end{bemerkung}

\subsection{Die Massenerhaltung}
\begin{bemerkung} Ein Erhaltungssatz

Sei $\omega$ ein beliebiges, offenes Teilgebiet von $\Omega$ mit genügend glatter Oberfläche $\partial \omega$, das konstant in der Zeit ist. Dann ist die Gesamtmasse in $\omega$ gegeben durch
\begin{align*}
  m(t) = \int_{\omega}\rho(t, x) \,dx
\end{align*}
(in [kg]).

Wenn die Masse in der zeitlichen Evolution in $\omega$ erhalten bleibt, so muss die zeitliche Rate der Massenänderung mit dem Massenfluss $\rho v(t, x)$ [kg/m$^{3}$s] über die Oberfläche von $\partial \omega$ übereinstimmen. Das heißt, die Änderung der Masse beträgt
\begin{align}\label{eq:mass_deriv}
  \frac{d}{dt} m(t) = \frac d {dt} \int_{\omega} \rho(t, x) dx = - \int_{\pd \omega} \rho v(t, x) \cdot n \,ds,
\end{align}
wobei $n(s)$ eine äußere Einheitsnormale bei $s \in \pd \omega$ ist. Da alle Funktionen als genügend glatt angenommen werden, kann der Gauß'sche Satz angewendet werden. 

Mit \eqref{eq:mass_deriv} führt dies auf
\begin{align*}
  \int_{\omega} \pd_{t}\rho(t, x) + \nabla\cdot(\rho v)\,dx = 0. 
\end{align*}
Da $\omega$ ein beliebiges Kontrollvolumen ist, folgt
\begin{align*}
  \pd_{t} \rho + \nabla \cdot(\rho v) = 0, \qquad \forall t \in [0, T], \, x \in \Omega
\end{align*}
(schöne Integralgleichung).
Diese Erhaltungsgleichung ist die erste Grundgleichung der Fluiddynamik., die (Massen-)Kontinuitätsgleichung.  
\end{bemerkung}

